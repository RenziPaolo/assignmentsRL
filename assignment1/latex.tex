\documentclass[10pt,a4paper]{article}

% Standard required packages
\usepackage[utf8]{inputenc}
\usepackage{amsmath,bm}
\usepackage{graphicx}
\usepackage{caption}
\usepackage{subcaption}
\usepackage{hyperref}
\usepackage{systeme}
\hypersetup{
    colorlinks,
    citecolor=black,
    filecolor=black,
    linkcolor=black,
    urlcolor=black
}

\begin{document}

\begin{titlepage}
    \begin{center}
        \vspace*{1cm}
        \Huge\textbf{Assignment 1}\\
        \vspace{1.5cm}
        \Large Author:
        \textbf{Paolo Renzi}\\
        \vspace{0.5cm}
        %\Contributors: \textbf{Leonardo Colosi, Francesco Bruno Nocera}%add contributors here
        \vfill
        \includegraphics[width=0.7\textwidth]{images/sapienza_logo.png}
        \vfill
        \vspace{0.8cm}
        \Large \textit{MARR, RL}\\
        \today
    \end{center}
\end{titlepage}
\newpage

\section*{Theory}
\subsection*{Problem 1}
we can derive how many steps are needed from this equation:
\begin{equation}  {2\gamma^i}/{(1-\gamma)}||Q0-Q*|| \leq \epsilon \end{equation}
because $ {2\gamma^i}/{(1-\gamma)}||Q0-Q*|| $ is the distance between $ v^{\pi i}(s) $ and $v^*(s) $. \newline
\newline
First of all we know both $Q0$ and $Q* \in [0, 1/(1-\gamma)]$ By the infinity norm, the maximum value is $ 1/(1-\gamma) $. 
\begin{equation}
{2\gamma^i}/{(1-\gamma)^2} \leq \epsilon
\end{equation}
Then we add and subtract 1 
\begin{equation}
    {2 (1 - (1 - \gamma))^i}/{(1-\gamma)^2} \leq \epsilon
\end{equation}
Then we substitute $ - (1 - \gamma) $ with $e^- (1 - \gamma)^2 $ because it makes solving it easier even if it makes the approsimation less accurate
\begin{equation}
    {2e^{-(1 - \gamma )i}}/{(1-\gamma)^2} \leq \epsilon
\end{equation}
Then we divide by 2 and multiply by $ {(1-\gamma)^2} $ and get 
\begin{equation}
e^{-(1 - \gamma )i} \leq  {(1-\gamma)^2}/2
\end{equation}
Then we take the log of both sides 
\begin{equation}
    -i(1 - \gamma ) \leq  -\log({2/(\epsilon(1-\gamma)^2))}
\end{equation}
In the end we divide by $ -(1 - \gamma ) $ (so changing also the $\leq$ in $\geq$)
\begin{equation}
    i \geq  \log({2/(\epsilon(1-\gamma)^2))}/(1 - \gamma ) 
\end{equation}

\newpage
\subsection*{Problem 2}
MDP \newline
S1, ..., S7
r(S,a)\begin{equation}
    \systeme{
      1/2 S = S_{1}
      5 S = S_{7}
      0 otherwhise
    }
\end{equation}
\newline
P(S_{6}|S_{6},a_{1}) = 0.3
P(S_{7}|S_{6},a_{1}) = 0.7
\newline
\pi(s) = a_{1} \forall s
v^1 = [0.5, 0, 0, 0, 0, 0, 5]
\gamma = 0.9

We calculate the discounted reward from S_{6} and add it to the current reward of s_{6} to get the value of s_{6}
V(S_{6}) = r(S_{6}) + \gamma \EX(V(S')) = \newline
= 0 + 0.9 (\mathbb{P}(S_{6}|S{6,a_{1}}) V(S_{6}) + \mathbb{P}(S_{7}|S{6,a_{1}}) V(S_{7}) ) = 0.9(0.3 *0 + 0.7 * 5 ) = 0.9 * 3.5 = 3.15 
\newpage
\section*{Code}

\subsection*{Policy iteration}
To implement the reward function I used an if to check if i was in the goal state and return 1 and an else to return 0 in all other cases
\newline
To check if the transition was feasibile I used a couple of ifs, one to check if it was trying to go out of the grid and the second to check if it's going against an obstacle 
\newline
To return the transition probability i first checked if the transition was feasibile and then I assigned 1/3 to each transition (if the transition isn't feasibile with the action it would would just remain the state it was)
\newpage
\subsection*{iLQR}
for iLQR i just implemented the formulas being careful about which operator i was using between * and @.

\end{document}