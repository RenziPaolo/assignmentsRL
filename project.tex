\documentclass[10pt,a4paper]{article}

% Standard required packages
\usepackage[utf8]{inputenc}
\usepackage{amsmath,bm}
\usepackage{graphicx}
\usepackage{caption}
\usepackage{subcaption}
\usepackage{hyperref}
\usepackage{systeme}
\usepackage{mathbbol}
\usepackage{mathtools}
\usepackage{biblatex}
\addbibresource{project.bib}
\hypersetup{
    colorlinks,
    citecolor=black,
    filecolor=black,
    linkcolor=black,
    urlcolor=black
}

\begin{document}

\begin{titlepage}
    \begin{center}
        \vspace*{1cm}
        \Huge\textbf{Project Proposal}\\
        \vspace{1.5cm}
        \Large Author:
        \textbf{Paolo Renzi}\\
        %\Large Contributors:\textbf{Bruno Francesco Nocera 1863075, Silverio Manganaro 1817504, Simone Tozzi, 1615930, Leonardo Colosi 1799057, Jacopo Tedeschi 1882789, Amine Ahardane 2050689.}
        \vspace{0.5cm}
        \vfill
        \includegraphics[width=0.7\textwidth]{images/sapienza_logo.png}
        \vfill
        \vspace{0.8cm}
        \Large \textit{MARR, RL}\\
        \today
    \end{center}
\end{titlepage}
\newpage
I will apply MuZero \cite{Schrittwieser_2020} to the game of StarCraft II, thanks to the environment PySC2\cite{vinyals2017starcraft}, starting from it's minigames, to the full game and compare it to AlphaStar \cite{Arulkumaran_2019}, Deepmind's agent. 
I will also try to use a different reward function then the one given by the environment, by scaling the ternary reward and summing it to a scaled version of "Blizzard score" by the time since the beginning of the game 

\vspace{300pt}

\printbibliography

\end{document}