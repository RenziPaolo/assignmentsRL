\documentclass[10pt,a4paper]{article}

% Standard required packages
\usepackage[utf8]{inputenc}
\usepackage{amsmath,bm}
\usepackage{graphicx}
\usepackage{caption}
\usepackage{subcaption}
\usepackage{hyperref}
\usepackage{systeme}
\usepackage{mathbbol}
\usepackage{mathtools}
\hypersetup{
    colorlinks,
    citecolor=black,
    filecolor=black,
    linkcolor=black,
    urlcolor=black
}

\begin{document}

\begin{titlepage}
    \begin{center}
        \vspace*{1cm}
        \Huge\textbf{Assignment 1}\\
        \vspace{1.5cm}
        \Large Author:
        \textbf{Paolo Renzi}\\
        \Large Contributors:\textbf{Bruno Francesco Nocera 1863075, Silverio Manganaro 1817504, Simone Tozzi, 1615930, Leonardo Colosi 1799057, Jacopo Tedeschi 1882789, Amine Ahardane 2050689.}
        \vspace{0.5cm}
        \vfill
        \includegraphics[width=0.7\textwidth]{images/sapienza_logo.png}
        \vfill
        \vspace{0.8cm}
        \Large \textit{MARR, RL}\\
        \today
    \end{center}
\end{titlepage}
\newpage

\section*{Theory}
\subsection*{Problem 1}

given the following Q table:
\vspace{5pt}
\begin{equation*} 
    Q(s,a) \:=\:
    \begin{pmatrix}
        Q(1,1) & Q(1,2)\\
        Q(2,1) & Q(2,2)
    \end{pmatrix}
    \:=\: 
    \begin{pmatrix}
        1 & 2\\
        3 & 4
    \end{pmatrix}
\end{equation*}
\vspace{5pt}


and this parameters:
\begin{align*}
    \alpha = 0.1 \\
    \gamma = 0.5
\end{align*}

and this experience:
\begin{equation*}
    (s,a,r,s') = (1,2,3,2)
\end{equation*}

Update equation for Q-Learning
\begin{equation*}
    Q(S, A) = Q(S, A) + \alpha [R +\gamma \max_a Q(S', a)- Q(S, A)]
\end{equation*}

\begin{equation*}
    Q(S, A) = 2 + 0.1 [3 + (0.5 * 4) - 2] = 2 + (0.1 * 3) = 2.3
\end{equation*}

for Sarsa we are given this next action
\begin{equation*}
    a' = \pi_\epsilon(s') = 2
\end{equation*}

Update equation for Sarsa 
\begin{equation*}
    Q(S, A) = Q(S, A) + \alpha [R +(\gamma Q(S', A'))- Q(S, A)]
\end{equation*}

\begin{equation*}
    Q(S, A) = 2 + 0.1 [3 + (0.5 *4)- 2] = 2 + (0.1 * 3) = 2.3
\end{equation*}

\newpage
\subsection*{Problem 2}

We want to prove this:
\begin{equation*}
    G_{t:t+n} - V_{t+n-1} (S_t) = \sum_{k = t}^{t+n-1} \gamma^k-t \delta_k  
\end{equation*}
knowing that:
\begin{equation*}
    G_{t:t+n} = R_{t+1} + \gamma R_{t+2} + \gamma^2 R_{t+3} + ... + \gamma^n-1 R_{t+n}+ \gamma^n V_{t+n-1} (S_{t+n}) 
\end{equation*}

So by subtracting $V(S_{t+1})$ to either sides i get

\begin{align*}
    G_{t:t+n} - V_{t+n-1} (S_t) = R_{t+1} + \gamma R_{t+2} + \gamma^2 R_{t+3} + ... \\
    + \gamma^n-1 R_{t+n}+ \gamma^n V_{t+n-1} (S_{t+n}) - V_{t+n-1} (S_t)
\end{align*}

If we assume that V will not change:

\begin{equation*}
    = R_{t+1} + \gamma R_{t+2} + \gamma^2 R_{t+3} + ... + \gamma^n-1 R_{t+n}+ \gamma^n V (S_{t+n}) - V (S_t)
\end{equation*}

\begin{equation*}
    = R_{t+1} + \gamma R_{t+2} + \gamma^2 R_{t+3} + ... + \gamma^n-1 R_{t+n}+ \gamma^n V (S_{t+n}) - V (S_t)
\end{equation*}

Recalling that $\delta_t = R_{t+1}(\gamma V(S_t+1) - V(s_t))$, we write $R_{t+1}$ in terms of $\delta_t$ 

\begin{equation*}
    = \delta_t - \gamma V(S_{t+1}) + V(S_{t}) + \gamma R_{t+2} + \gamma^2 R_{t+3} + ... + \gamma^{n-1} R_{t+n}+ \gamma^n V (S_{t+n}) - V (S_t) 
\end{equation*}

Iterate that substitution to all the rewards

\begin{align*}
    = \gamma^0 [\delta_t - \gamma V(S_{t+1}) + V(S_{t})] + \gamma[\delta_{t+1} - \gamma V(S_{t+2}) + V(S_{t+1})] + ... \\ \newline
    + \gamma^{n-1}[\delta_{t+n} - \gamma V(S_{t+n-1}) + V(S_{t})] + \gamma^n V (S_{t+n}) - V (S_t)            
\end{align*}

Write everything in terms of a summation

\begin{equation*}
    =  \sum_{k = t}^{t+n-1} [\gamma^{k-t}\delta_{k} - \gamma^{k-t+1} V(S_{k+1}) +\gamma^{k-t} V(S_{k})] + \gamma^n V (S_{t+n}) - V (S_t) 
\end{equation*}

Split the summation in it's components

\begin{equation*}
    =  \sum_{k = t}^{t+n-1} [\gamma^{k-t}\delta_{k}] -\sum_{k = t}^{t+n-2} [\gamma^{k-t+1} V(S_{k+1})] + \sum_{k = t+1}^{t+n-1}[\gamma^{k-t} V(S_{k})] 
\end{equation*}

Adjust the indices

\begin{equation*}
    =  \sum_{k = t}^{t+n-1} [\gamma^{k-t}\delta_{k}] -\sum_{k = t+1}^{t+n-1} [\gamma^{k-t} V(S_{k})] + \sum_{k = t+1}^{t+n-1}[\gamma^{k-t} V(S_{k})] 
\end{equation*}

We get what we wanted to demonstrate

\begin{equation*}
    =  \sum_{k = t}^{t+n-1} \gamma^k-t \delta_k  
\end{equation*}

\newpage
\section*{Code}

\subsection*{Sarsa($\lambda$) }
In Sarsa($\lambda$) I had to implement 2 things:
\begin{itemize}
 \item The $\epsilon$ greedy policy :
 I did so with an if-then-else, a random number and
 the max over the Q fuction for the greedy part and sampling randomly the action space
 for the exploration part
 \item The update step: I did so by implementig the pseudo-code on the slides (pack 7 slide 71) in particular
 the equations, being careful on when to use the matrix notation
\end{itemize}

\subsection*{Q-Learning TD($\lambda$) with RBF}
In Q-Learning TD($\lambda$) with RBF I had to implement 2 things:
\begin{itemize}
    \item The RBF:
    I did so first trying to implement it myself, but with not so great results,
    then i tried to use the sklearn implementation but i had to usa also a learned scaler to 
    have good performance, probably because otherwise it would have given too much importance
    to states that didn't need it
    \item The update step of Q-Learning TD($\lambda$)  : I did so by implementig the pseudo-code on the slides (pack 8 slide 29,) in particular
    the equations, being careful on when to use the matrix notation
   \end{itemize}

\end{document}